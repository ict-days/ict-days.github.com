\section{Theorie und Hinweise zu BASIC und BASIC Stamp}

\subsection{Was ist BASIC?}
\verb|BASIC| steht für 
\emph{Beginner's All-purpose Symbolic Instruction Code}
und bedeutet so viel wie
\emph{symbolische Allzweck-Programmiersprache für Anfänger}.
Diese Programmiersprache wurde 1964 entwickelt und bis heute 
ständig erweitert. Das \verb|BASIC Stamp| verwendet einen neueren
Dialekt dieser Programmiersprache. D.h. es ist sehr ähnlich aber 
nicht genau gleich.

\begin{lstlisting}[label=Pure BASIC Code, caption=Echter BASIC Code]
10  INPUT "Geben Sie bitte Ihren Namen ein"; A$
20  PRINT "Guten Tag, "; A$
30  INPUT "Wie viele Sterne möchten Sie?"; S
35  S$ = ""
40  FOR I = 1 TO S
50  S$ = S$ + "*"
55  NEXT I
60  PRINT S$
70  INPUT "Möchten Sie noch mehr Sterne?"; Q$
80  IF LEN(Q$) = 0 THEN GOTO 70
90  L$ = LEFT$(Q$, 1)
100 IF (L$ = "J") OR (L$ = "j") THEN GOTO 30
110 PRINT "Auf Wiedersehen";
120 FOR I = 1 TO 200
130 PRINT A$; " ";
140 NEXT I
150 PRINT
\end{lstlisting}

\noindent
Berühmt wurde diese Programmiersprache unter anderem durch den C64,
ein Computer aus dem Jahre 1984. Dieser hatte kein Betriebssystem,
wie Sie das auf Ihrem Computer kennen und gewohnt sind 
(Windows, GNU/Linux, MacOS, \dots), sondern er hatte einen 
\verb|BASIC|-Interpreter. D.h. man konnte "programmieren". 

Wer gerne mehr über die Programmiersprache \verb|BASIC| oder den C64
erfahren möchte,
findet auf der freien Enzyklopädie Wikipedia gute Informationen
(siehe \url{http://de.wikipedia.org/wiki/BASIC} und 
\url{http://de.wikipedia.org/wiki/C64}).

\subsection{Entwicklungsumgebung}
Die Entwicklungsumgebung, die zur Programmierung des \verb|BASIC Stamp|
benötigt wird, kann auf der Homepage des Herstellers bezogen werden auf
\begin{center}
\url{http://www.parallax.com}
\end{center}

\noindent
Auf dieser Seite navigieren Sie zu \verb|Downloads > BASIC Stamp Software|.
Dort finden Sie die Software für Windows, Macintosh und Linux.

Auf der Homepage finden Sie auch weitere Informationen zur Programmiersprache.

\newpage
\subsection{Instruction Set}
Wie jede Programmiersprache hat auch \verb|BASIC| ein sogenanntes Instruction Set.
Diese zeigt auf, welche Befehle es in der Programmiersprache gibt und wie man 
diese einsetzt. Der Hersteller \url{www.parallax.com} bietet hierzu eine grosse
Dokumentation an. Wir möchten hier nur auflisten, was für uns im Kurs wichtig ist.

\begin{table}[h!]
\centering

\begin{tabular}{| l || l | p{4cm} | }
\rowcolor[gray]{0.5}
\hline
	\textbf{Befehl} 
	& \textbf{Beispiel} 
	& \textbf{Beschreibung} \\
\hline
\hline
	\verb|END| 
	& \verb|END| 
	& \footnotesize{Ende/Bei Reset startet das Programm neu.} \\ 
\hline
\rowcolor[gray]{0.8}
	\verb|FOR| \dots \verb|NEXT| 
	& \verb|FOR B2 = 0 TO 255| \dots \verb|NEXT| 
	& \footnotesize{Zählt von 0 bis 255, erst dann geht es zur Zeile nach dem \verb|NEXT|.} \\ 
\hline
	\verb|GOSUB| 
	& \verb|GOSUB label| \dots \verb|RETURN|
	& \footnotesize{\verb|GOSUB|: Springe zu \verb|label|. \verb|RETURN|: Springe zurück zur Zeile nach GOSUB.} \\
\hline
\rowcolor[gray]{0.8}
	\verb|GOTO| 
	& \verb|GOTO label| 
	& \footnotesize{Springe zu \verb|label|.} \\
\hline
	\verb|HIGH|
	& \verb|HIGH 3| 
	& \footnotesize{Setze Pin 3 auf logisch 1 (d.h. einschalten).} \\
\hline
\rowcolor[gray]{0.8}
	\verb|IF| \dots \verb|THEN|
	& \verb|IF B2 = 1 THEN label|
	& \footnotesize{Falls B2 gleich 1 ist, dann springe zu \verb|label|.} \\
\hline
	\verb|INPUT| 
	& \verb|INPUT 5| 
	& \footnotesize{Pin 5 als Eingabe definieren.} \\
\hline
\rowcolor[gray]{0.8}
	\verb|LOW|
	& \verb|LOW 3| 
	& \footnotesize{Setze Pin 3 auf logisch 0 (d.h. ausschalten).} \\
\hline
	\verb|OUTPUT| 
	& \verb|OUTPUT 3| 
	& \footnotesize{Pin 3 als Ausgabe definieren.} \\
\hline
\rowcolor[gray]{0.8}
	\verb|PAUSE| 
	& \verb|PAUSE 100| 
	& \footnotesize{Macht nichts für 100 Millisekunden.} \\
\hline
	\verb|TOGGLE|
	& \verb|TOGGLE 5| 
	& \footnotesize{Invertiere Pin 5. Invertieren heisst umkehren, d.h. aus 0 wird 1 und aus 1 wird 0.} \\
\hline
\end{tabular} 
\end{table}

\noindent
Wer gerne auch nach dem Kurs Programme schreiben möchte, kann beim Instruktor 
nach einem grösseren Instruction Set nachfragen.

\newpage
\subsection{Platzhalter}\label{Platzhalter}
Platzhalter oder Variabeln vereinfachen die Programmierung, denn Sie
erlauben es, bestimmten Dingen eigene Namen zu geben usw.

Sie haben in der Aufgabe aus \ref{timing-problem}
gelernt, dass z.B. hinter der Bezeichnung \lstinline{PIN0} der Taster
steht. In Ihrem Programm muss dieser nicht zwingend \lstinline{PIN0}
heissen, denn Sie können einen eigenen Namen für diesen erstellen
z.B. \lstinline{Taste0}.

Um solche Variablen zu erstellen, braucht es Speicher (RAM).
Leider gibt es beim \verb!BASIC Stamp! nicht sehr viel davon sondern
nur die folgenden Speicherstellen:

\begin{table}[h!]
\centering
\begin{tabular}{lll}
\textbf{Wort} & \textbf{Byte} & \textbf{Bit} \\
\hline 
&&\\
W0 	& B0 	& Bit 0--7 		\\
	& B1 	& Bit 8--15 	\\
&&\\
W1	& B2	& Bit 0--7 		\\
	& B3	& Bit 8--15 	\\
&&\\
W2 	& B4 	& Bit 0--7 		\\
	& B5 	& Bit 8--15 	\\
&&\\
W3	& B6	& Bit 0--7 		\\
	& B7	& Bit 8--15 	\\
&&\\
W4 	& B8 	& Bit 0--7 		\\
	& B9 	& Bit 8--15 	\\
&&\\
W5	& B10	& Bit 0--7 		\\
	& B11	& Bit 8--15 	\\
\end{tabular}
\caption{RAM-Übersicht für Variablen}
\end{table}

\noindent
Lassen Sie sich nicht einschüchtern oder verwirren durch die obige Tabelle.
Diese zeigt lediglich auf, dass das \verb|BASIC Stamp| in Worten speichert,
in denen jeweils zwei Byte liegen. Ein solches Byte hat jeweils 8 Bit.

\[ \text{ein Wort} \quad 
   = \quad \text{zwei Byte} \quad
   = \quad 2 \cdot (\text{8 Bit}) \quad
   = \quad 2^{16} \quad
   = \quad 65536 \]

\noindent
Beim arbeiten mit diesem Speicher muss man sich entscheiden, was man verwenden
möchte. Benutzt man das Wort \lstinline{W0} so kann man nicht auch noch die
Bytes \lstinline{B0} und \lstinline{B1} verwenden, da diesen ja im Wort
\lstinline{W0} enthalten sind. Dies gilt natürlich für alle Speicherplätze.