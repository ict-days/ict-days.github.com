\section{Lösungen}


\ifteacher
\subsection{Aufgabe 1}
\begin{enumerate}[label=(\alph*)]
	\item Die einzelnen Zeilen im Programm sind jeweils Dokumentiert
	mit einem Kommentar. Kommentare erkennt man daran, dass diese mit
	einem \lstinline{'} eingeleitet und in der Entwicklungsumgebung
	grün dargestellt werden. Alles was rechts von diesem 
	Zeichen steht, ist ein Kommentar und wird vom Computer ignoriert.\\
	\textbf{Beispiel:} In der Zeile 3 steht der Kommentar 
	\lstinline{'LED 0 einschalten}.
	\begin{itemize}
		\item Als erstes wird mittels \lstinline{HIGH 0} die LED 0 
		eingeschaltet.
		\item Danach wird eine Pause von 500 Millisekunden 
		(\nicefrac{1}{2} Sekunde) mit \lstinline{PAUSE 500} eingelegt.
		\item Mit \lstinline{LOW 0} wird die LED 0 wieder ausgeschaltet.
		\item Erneut wird wieder mit \lstinline{PAUSE 500} eine halbe
		Sekunde lang nichts gemacht.
		\item Mit \lstinline{GOTO start} wird als nächstes die Zeile
		im Programm ausgeführt wo das Label \lstinline{start} steht
		(d.h. in Zeile 3). Durch das \lstinline{GOTO start} wird das
		Programm unendlich oft wiederholt, da es am Ende immer an den
		Anfang (\lstinline{start}) springt.
	\end{itemize}
	\item Um das Programm langsamer zu machen, muss die Pausenzeit
	verlängert werden. Man könnte diese z.B. auf 1000ms setzen.
	\lstinputlisting[label=, caption=]{../listings/test-slow.bas}
	\item Um eine andere LED blinken zu lassen muss die Nummer hinter 
	\lstinline{HIGH} und \lstinline{LOW} geändert werden, z.B. auf 7.
	\lstinputlisting[label=,caption=]{../listings/test-other.bas}
\end{enumerate}
\fi

\ifteacher
\newpage
\subsection{Aufgabe 2}
\begin{enumerate}[label=(\alph*)]
	\item Das Programm schaltet eine LED nach der anderen ein und wieder
	aus. Es beginnt dabei bei der LED 0 und nach der LED 7 macht es von
	vorne weiter.
	\item ~
	\item Um eine LED nach der anderen ein- und ausschalten zu lassen 
	von der LED 7 beginnend zur LED 0 müsste wie folgt aussehen.
	\lstinputlisting[label=,caption=]{../listings/halfknight-reverse.bas}
\end{enumerate}
\fi

\ifteacher
\newpage
\subsection{Aufgabe 3}
\begin{enumerate}[label=(\alph*)]
	\item Wir können das Programm aus der vorhergehenden Aufgabe nehmen
	und diese einfach erweitern.
	\lstinputlisting[label=,caption=]{../listings/knightrider-simple.bas}
\end{enumerate}
\fi

\ifteacher
\newpage
\subsection{Aufgabe 4}
\begin{enumerate}[label=(\alph*)]
	\item Um das Programm schneller zu machen muss lediglich die Pausenzeit
	kleiner gewählt werden. Zum Beispiel auf 100ms mit
	\lstinline{PAUSE 100}.
	\item Ihnen sollte aufgefallen sein, dass die Veränderung der Pausenzeit
	eine einfache Aufgabe ist, jedoch sehr mühsam da man diese an vielen 
	Stellen im Programm von Hand ändern muss.
\end{enumerate}
\fi

\ifteacher
\newpage
\subsection{Aufgabe 5}
\begin{enumerate}[label=(\alph*)]
	\item Die Taste 0 wird als Eingabe definiert. Ist diese gedrückt
	beim Label \lstinline{start} so springt das Programm zum Label 
	\lstinline{abc}. Dort angelangt wird die LED 4 eingeschaltet mit
	\lstinline{HIGH 4}. Solange die Taste gedrückt ist, bleibt die
	LED 4 eingeschaltet, denn es springt zu \lstinline{start} und
	gleich danach zu \lstinline{abc}. Lässt man die Taste nun los
	geht es nicht zu \lstinline{abc} sondern einfach zur nächsten Zeile
	im Programm. Dort steht dann \lstinline{LOW 4} was die LED 4
	ausschaltet.

	\textbf{Zusammengefasst:}\\
	Drückt man die Taste 0, so leuchtet die LED 4.
	Lässt man sie los, so leuchtet sie nicht.
	\item Drückt man eine Taste, so leuchtet immer die zugehörige LED.
	Das hat nichts mit dem Programm zu tun sondern ist etwas spezielles
	am Board selbst.
\end{enumerate}
\fi

\ifteacher
\newpage
\subsection{Aufgabe 6}
\begin{enumerate}[label=(\alph*)]
	\item Einerseits müssen wir mehr Taster definieren und andererseits
	müssen wir mehr Sprungsstellen definieren.
	\lstinputlisting[label=,caption=]{../listings/button-quad.bas}
	\item Es muss zwei mal nach einander angefragt werden ob die Tasten
	gedrückt sind. Zudem können die Tasten in verschiedenen Reihenfolgen
	gedrückt werden.
	\lstinputlisting[label=,caption=]{../listings/button-double.bas}
	\item Es gibt insgesamt 16 Kombinationen, denn man hat vier Taster
	mit jeweils zwei Zuständen (Ein/Aus) und somit 
	$2^4=2 \cdot 2 \cdot 2 \cdot 2=16$
	Kombinationen.
\end{enumerate}
\fi

\ifteacher
\newpage
\subsection{Aufgabe 7}
\begin{enumerate}[label=(\alph*)]
\item \dots
\item \dots 
\item Ein Zyklus soll 15 Sekunden dauern. In solch einen Zyklus soll eine gerade
Anzahl Pausen mit 300 Millisekunden stattfinden. Cooles Problem denn wir können 
einfach Algebra verwenden:
\[  \text{15 Sekunden} = x \cdot 2 \cdot \text{Pausenzeit} \]
\[  15 = x \cdot 2 \cdot 0.3 \qquad |\div 2 \]
\[  \frac{15}{2} = x \cdot 0.3 \qquad |\div 0.3 \]
\[ \frac{15}{2 \cdot 0.3} = x  \qquad |\Leftrightarrow x \]
\[ x = \frac{15}{2 \cdot 0.3} \]
\[ x = \frac{15}{0.6} \]
\[ x = 25 \]
Nun wissen wir, dass wir 25 Zyklen brauchen. Da wir bei 0 und nicht bei 1 zu Zählen 
beginnen, müssen wir 1 abziehen d.h. wie schreiben \lstinline{FOR B2 = 0 TO 25}.
\lstinputlisting[label=,caption=]{../listings/counter-blink.bas}
\end{enumerate}
\fi

\ifteacher
\newpage
\subsection{Aufgabe 8}
\begin{enumerate}[label=(\alph*)]
\item \dots
\item \dots
\item \dots
\end{enumerate}
\fi

\ifteacher
\newpage
\subsection{Aufgabe 9}
\begin{enumerate}[label=(\alph*)]
\item \dots
\item \dots
\item Code \\ \lstinputlisting[label=,caption=]{../listings/better-knight.bas}
\item Code \\ \lstinputlisting[label=,caption=]{../listings/double-knight.bas}
\end{enumerate}
\fi